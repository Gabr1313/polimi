\documentclass[12pt, a4paper]{article}

\usepackage[margin=2cm]{geometry}
\usepackage{ragged2e}
\usepackage{amssymb}
\usepackage{amsmath}
\usepackage{amsthm}
\usepackage{hyperref}
\usepackage{thmtools}
\usepackage[utf8]{inputenc}
\usepackage{tocloft}

\renewcommand{\listtheoremname}{Indice}
\newtheoremstyle{break}
	{}           %Space above, empty = `usual valuè
	{}           %Space below
	{}           %Body font
	{}           %Indent amount (empty = no indent, \parindent = para indent)
	{\bfseries}  %Thm head font
	{.}          %Punctuation after thm head
	{\newline}   %Space after thm head: \newline = linebreak
	{}           %Thm head spec
\renewcommand{\proofname}{Dimostrazione}
\theoremstyle{break}
\newtheorem{theorem}{Teorema} %[section]

\title{Dimostrazioni}
\author{Gabr1313}
\date{\today}

\begin{document}
\pagenumbering{roman}
\justify
\sloppy
\maketitle
\listoftheorems

\newpage
\pagenumbering{arabic}
\begin{theorem} [Formula risolutiva per le EDO del primo ordine lineari]
	Date $a, b: I \subseteq \mathbb{R} \to \mathbb{R}$ continue,
	l'integrale generale delle EDO
	\[
		y'(t) = a(t) y(t) + b(t)
	\]
	con $t \in I $, è dato da:
	\[
		y(t) = e^A \left( \int e^{-A} b(t) dt + c \right)
	\]
	dove $A = \int a(t) dt$, $t \in I$, $c \in \mathbb{R}$.
\end{theorem}
\begin{proof} Calcoli e teorema fondamentale del calcolo integrale
	\begin{itemize}
		\item Derivando la seguente funzione, si ottine
			\begin{align*}
				(y e^{-A})' & = y' e^{-A} + [ye^{-A}(-a)] \\
				            & = y' e^{-A} - aye^{-A}       \\
				            & = (y' - ay)e^{-A}            \\
				            & = (b)e^{-A}
			\end{align*}
		\item Utilizzando il teorema fondamentale del calcolo integrale
			\begin{align*}
				y e^{-A} & = \int (y e^{-A})' dt       \\
				         & = \int be^{-A} dt + c     \\
				y        & = e^A \int be^{-A} dt + c \\
			\end{align*}
			dove $c \in \mathbb{R}$.
	\end{itemize}
\end{proof}

\newpage
\begin{theorem} [Teorema di struttura dell’integrale generale di equazioni del
	secondo ordine lineari omogenee]
	Siano $a, b, c: I \subseteq \mathbb{R} \to \mathbb{R}$ continue, $a
		\neq 0$, allora la EDO
	\[
		a(t) y''(t) + b(t) y'(t) + c(t) y(t) = 0
	\]
	con $t \in I $ ha come IG uno spazio vettoriale di dimensione 2
	\[
		y_o(t) = c_1 y_{o,1}(t) + c_2 y_{o,2}(t)
	\]
	con $c_1, c_2 \in \mathbb{R}$, dove $y_{o,1}, y_{o,2}$ sono due soluzioni
	linearmente indipendenti dell'EDO.
\end{theorem}
\begin{proof} Principio di Sovrapposizione + $\exists !$ del PC
	\begin{itemize}
		\item per il pricipio di Sovrapposizione l'insieme di tutte le soluzioni
			è uno spazio vettoriale, perchè chiuso rispetto alla somma e al
			prodotto per uno scalare. Rimane quindi da dimostrare che la
			dimesione di tale spazio vettoriale è 2:
			\begin{itemize}
				\item si trovano $y_{o,1}$ e $y_{o,2}$ LI
				\item si dimostra che ogni soluzione si possa scrivere come
					combinazione lineare di queste due
			\end{itemize}
		\item Si scelgono $y_{o,1}, y_{o,2}$ come soluzioni di problemi di
			Cauchy seguenti con $t_o \in I$ fissato
			\begin{align*}
				\begin{cases}
					a(t) y''(t) + b(t) y'(t) + c(t) y(t) = 0 \\
					y(t_o) = 1                               \\
					y'(t_o) = 0                              \\
				\end{cases}
			\end{align*}
			\begin{align*}
				\begin{cases}
					a(t) y''(t) + b(t) y'(t) + c(t) y(t) = 0 \\
					y(t_o) = 0                               \\
					y'(t_o) = 1                              \\
				\end{cases}
			\end{align*}
			Queste soluzioni esistono e sono uniche per il teorema di $\exists
				!$ del PC. $y_{o,1}$ e $y_{o,2}$ sono inoltre LI perchè se per
			assurdo si ipotizzasse che $\exists c \neq 0 : y_{o,1} = c y_{o,2}
				\quad \forall t \in I$, allora $1 = y_{o,1}(t_o) = c y_{o,2}
				(t_o) = 0$.
		\item Sia $\bar{y}_o$ una qualunque soluzione particolare della EDO, si
			cercano $c_1, c_2 \in \mathbb{R}$ in modo tale che
			$\bar{y}_o(t)$ coincida con
			\[
				z(t) = c_1 y_{o,1}(t) + c_2 y_{o,2}(t)
			\]
			la quale è una soluzione per il Principio di Sovrapposizione.\\
			Si scelgono quindi $c_1, c_2$ in modo che:
			\begin{align*}
				\bar{y}_o(t_o) = z(t_o) = c_1 y_{o,1}(t_o) + c_2 y_{o,2}(t_o) =
				c_1         \\
				\bar{y}'_o(t_o) = z'(t_o) = c_1 y'_{o,1}(t_o) + c_2 y'_{o,2}
				(t_o) = c_2 \\
			\end{align*}
			quindi
			\begin{align*}
				z(t) = \bar{y}_o(t_o) y_{o,1}(t) + \bar{y}'_o(t_o) y_{o,2}(t)
			\end{align*}
			Sia $\bar{y}_o$ che $z$ soddisfano lo stesso PC
			\begin{align*}
				\begin{cases}
					a(t) y''(t) + b(t) y'(t) + c(t) y(t) = 0 \\
					y(t_o) = y_o(t_o)                        \\
					y'(t_o) = y_o'(t_o)                      \\
				\end{cases}
			\end{align*}
			e quindi per il teorema di $\exists !$ del PC $\bar{y}_o(t) = z(t)$.
	\end{itemize}
\end{proof}

\newpage
\begin{theorem} [Criterio della radice e criterio del rapporto per la
	determinazione del raggio di convergenza]
	Data la serie
	\[
		\sum_{n=0}^{\infty} a_n (x - x_o)^n
	\]
	con $a_n \in \mathbb{R}$, capita uno dei seguenti casi:
	\[
		\exists R = \lim_{n \to \infty} \left| \frac{a_n}{a_{n+1}}
		\right|   \quad \text{oppure} \quad
		\exists R = \lim_{n \to \infty} \left| \frac{1}{\sqrt[n]{a_n}}
		\right|
	\]
	con $R \in [0,+\infty]$, allora la serie converge e ha raggio $R$.
\end{theorem}
\begin{proof} Applicazione dei criteri del rapporto e della radice
	\begin{itemize}
		\item Per il teorema del raggio di convergenza di una serie di potenze
			è sufficiente verificare che:
			\[
				\sum_{n=0}^{\infty} \left| a_n (x - x_o)^n \right|
			\]
			\begin{itemize}
				\item converge per $|x - x_o| < R$
				\item non converge per $|x - x_o| > R$
			\end{itemize}
		\item tutte le serie convergono per $x = x_o$; nel caso in cui $x \neq
				x_o$ si può applicare il criterio del rapporto alla serie di
			partenza:
			\[
				\lim_{n \to \infty} \left| \frac{a_{n+1}(x-x_0)^{n+1}}
				{a_n (x-x_0)^n} \right| = \left| x - x_0 \right| \lim_{n
				\to \infty} \left| \frac{a_{n+1}}{a_n} \right| =
				\frac{\left| x -
				x_0 \right|}{R}
			\]
			E quindi i 2 punti precedenti sono verificati.
		\item $\quad \forall x \in \mathbb{R} $ si può applicare il criterio
			della radice alla serie di partenza:
			\[
				\lim_{n \to \infty} \sqrt[n]{\left| a_n (x-x_0)^n
				\right|} = \lim_{n \to \infty} \left| x - x_0 \right|
				\sqrt[n]{\left| a_n \right|} = \frac{\left| x - x_0 \right|}{R}
			\]
			E quindi i 2 punti precedenti sono verificati.
	\end{itemize}
\end{proof}

\newpage
\begin{theorem} [Calcolo dei coefficienti di Fourier di una funzione periodica]
	Se $f: \mathbb{R} \to \mathbb{R}$ è una funzione $2\pi$-periodica e
	somma di una funzione trigonometrica:
	\[
		f(x) = a_0 + \sum_{n=1}^{\infty} (a_n \cos(nx) + b_n \sin(nx))
	\]
	con $a_n, b_n \in \mathbb{R}$ e convergenza totale in $[-\pi,\pi]$,
	allora $a_n, b_n$ sono i coefficienti di Fourier di $f$:
	\begin{align*}
		a_0 & = \frac{1}{2\pi} \int_{-\pi}^{\pi} f(x) dx         \\
		a_n & = \frac{1}{\pi} \int_{-\pi}^{\pi} f(x) \cos(nx) dx \\
		b_n & = \frac{1}{\pi} \int_{-\pi}^{\pi} f(x) \sin(nx) dx \\
	\end{align*}
\end{theorem}
\begin{proof} calcoli utilizzando le formule di ortogonalità
	\begin{itemize}
		\item $a_0$
			\begin{align*}
				        & \frac{1}{2\pi} \int_{-\pi}^{\pi} f(x) dx =           \\
				= \quad & \frac{1}{2\pi} \int_{-\pi}^{\pi} \left[a_0 +
				\sum_{n=1}^{\infty} (a_n \cos(nx) + b_n \sin(nx))\right] dx  = \\
				\intertext{per convergenza totale}
				= \quad & \frac{1}{2\pi} \left( a_0 \int_{-\pi}^{\pi} dx +
				\sum_{n=1}^{\infty} a_n \int_{-\pi}^{\pi} \cos(nx) dx +
				\sum_{n=1}^{\infty} b_n \int_{-\pi}^{\pi} \sin(nx) dx \right)
				=                                                              \\
				= \quad & \frac{1}{2\pi} \left( 2\pi a_0 + 0 + 0 \right)   =   \\
				= \quad & a_0
			\end{align*}
		\item $a_n$
			\begin{align*}
				        & \frac{1}{\pi} \int_{-\pi}^{\pi} f(x) \cos(nx) dx =       \\
				= \quad & \frac{1}{\pi} \int_{-\pi}^{\pi} \left[a_0 +
				\sum_{n=1}^{\infty} (a_n \cos(nx) + b_n \sin(nx))\right]
				\cos(nx) dx =                                                      \\
				\intertext{per convergenza totale: $| \cos(nx) | \leq 1$ }
				= \quad & \frac{1}{\pi} \left( a_0 \int_{-\pi}^{\pi} \cos(nx) dx +
				\sum_{n=1}^{\infty} a_n \int_{-\pi}^{\pi} \cos^2(nx) dx +
				\sum_{n=1}^{\infty} b_n \int_{-\pi}^{\pi} \cos(nx) \sin(nx)
				\right)                                                 =          \\
				\intertext{per le formule di ortogonalità delle serie armoniche
				n-esime}
				= \quad & \frac{1}{\pi} \left( 0 + a_n \pi + 0 \right)  =          \\
				= \quad & a_n
			\end{align*}
		\item $b_n$ si dimostra in modo analogo a $a_n$.
	\end{itemize}
\end{proof}

\newpage
\begin{theorem} [Invarianza della lunghezza per riparametrizzazioni]
	Sia $r: [a,b] \subseteq \mathbb{R} \to \mathbb{R}^n$ una curva
	regolare con sostegno $\gamma$ e sia $v: [c,d] \subseteq \mathbb{R}
		\to \mathbb{R}^n$ una riparametrizzazione di $\gamma$ relativa
	alla variabile $ \varphi : [c,d] \to [a,b]$ (monotona), cioè
	$v(s) = r(\varphi (s)) \quad \forall s \in [c,d]$, allora
	\[
		\text{len}(r([a,b])) = \int_{c}^{d} || v'(s) || ds =
		\text{len}(v([c,d])) \]
\end{theorem}
\begin{proof} calcoli e discussione della monotonia di $\varphi$
	\begin{itemize}
		\item Siccome r è regolare
			\[
				\text{len}(r([a,b])) = \int_{a}^{b} || r'(t) || dt
			\]
		\item allora $\forall s \in [c,d]$
			\[
				v'(s) =
				\begin{bmatrix}
					v_1'(s) \\
					\vdots  \\
					v_n'(s) \\
				\end{bmatrix}
				=
				\begin{bmatrix}
					[r_1(\varphi(s))]' \\
					\vdots             \\
					[r_n(\varphi(s))]' \\
				\end{bmatrix}
				=
				\begin{bmatrix}
					r_1'(\varphi(s))\varphi(s)' \\
					\vdots                      \\
					r_n'(\varphi(s))\varphi(s)' \\
				\end{bmatrix}
				= \varphi'(s) r'(\varphi(s))\\
			\]
			quindi
			\[
				||v'(s)|| = ||\varphi'(s) r'(\varphi(s))|| = |\varphi'(s)| ||
				r'(\varphi(s))||
			\]
		\item sia $t = \varphi(s)$, allora $ dt = \varphi'(s) ds$
		\item se $\varphi$ è crescente, allora $\varphi(c) = a < b =
				\varphi(d)$ e $\varphi'(s) \geq 0 \quad \forall s \in [c,d]$,
			quindi
			\[
				||v'(s)|| = \varphi'(s) || r'(\varphi(s))||
			\]
			\[
				\text{len}(r([a,b])) =
				\int_{a}^{b} || r'(t) || dt =
				\int_{c}^{d} || r'(\varphi(s))|| \varphi'(s) ds =
				\int_{c}^{d} || v'(s) || ds =
				\text{len}(v([c,d]))
			\]
		\item se $\varphi$ è decrescente, allora $\varphi(c) = b > a =
				\varphi(d)$ e $\varphi'(s) \leq 0 \quad \forall s \in [c,d]$,
			quindi
			\[
				||v'(s)|| = -\varphi'(s) || r'(\varphi(s))||
			\]
			\[
				\text{len}(r([a,b])) =
				\int_{a}^{b} || r'(t) || dt =
				\int_{d}^{c} || r'(\varphi(s))|| (-\varphi'(s)) ds =
				\int_{c}^{d} || v'(s) || ds =
				\text{len}(v([c,d]))
			\]
	\end{itemize}
\end{proof}

\newpage
\begin{theorem} [Differenziabilità implica continuità]
	Sia $A \subseteq \mathbb{R}^n$ aperto e $f: A \to \mathbb{R}$
	differenziabile in $\underline{x}_0 \in A$, allora $f$ è continua in
	$\underline{x}_0$.
\end{theorem}
\begin{proof} definizione, disuguaglianza triangolare, disuguaglianza di
	Cauchy-Schwarz, limite e teorema dei carabinieri
	\begin{itemize}
		\item Siccome $f$ è differenziabile in $\underline{x}_0$, allora
			\[
				f(\underline{x}) = f(\underline{x}_0) + \nabla f
				(\underline{x}_0) \cdot (\underline{x} - \underline{x}_0)
				+ o(||\underline{x} - \underline{x}_0||)
			\]
			ovvero
			\begin{align*}
				0 & \leq |f(\underline{x}) - f(\underline{x}_0)|               \\
				  & = |\nabla f
				(\underline{x}_0) \cdot (\underline{x} - \underline{x}_0) +
				o(||\underline{x} - \underline{x}_0||)|                        \\
				\intertext{per la disuguaglianza triangolare}
				  & \leq |\nabla f (\underline{x}_0) \cdot (\underline{x} -
				\underline{x}_0)| + |o(||\underline{x} - \underline{x}_0||)|   \\
				  & = |\nabla f (\underline{x}_0) \cdot (\underline{x} -
				\underline{x}_0)| + o(||\underline{x} - \underline{x}_0||)
				\intertext{per la disuguaglianza di Cauchy-Schwarz}
				  & \leq ||\nabla f (\underline{x}_0)|| \cdot ||(\underline{x}
				- \underline{x}_0)|| + o(||\underline{x} - \underline{x}_0||)
			\end{align*}
		\item passando ora al limite
			\begin{align*}
				0 & \leq \lim_{\underline{x} \to \underline{x}_0}
				f(\underline{x}) - f(\underline{x}_0)                           \\
				  & \leq \lim_{\underline{x} \to \underline{x}_0} \left(
				||\nabla f (\underline{x}_0)|| \cdot ||(\underline{x} -
				\underline{x}_0)|| \right) + \lim_{\underline{x} \to
				\underline{x}_0} o(||\underline{x} - \underline{x}_0||)         \\
				  & = ||\nabla f (\underline{x}_0)|| \cdot  \lim_{\underline{x}
				\to \underline{x}_0} ||(\underline{x} - \underline{x}_0)|| +
				\lim_{\underline{x} \to \underline{x}_0} \left(
				\frac{o(||\underline{x} - \underline{x}_0||)}
					{||\underline{x} - \underline{x}_o||}||\underline{x} -
				\underline{x}_o|| \right)                                       \\
				  & = 0
			\end{align*}
		\item quindi per il teormema dei carabinieri
			\[
				\lim_{\underline{x} \to \underline{x}_0} | f(\underline{x}) -
				f(\underline{x}_0) | = 0
			\]
			ovvero $f$ è continua in $\underline{x}_0$.
	\end{itemize}
\end{proof}

\newpage
\begin{theorem} [Formula del gradiente]
	Sia $A \subseteq \mathbb{R}^n$ aperto e $f: A \to \mathbb{R}$
	differenziale in $\underline{x}_0 \in A$, allora $\forall \underline{v} \in
		\mathbb{R}^n$, $||\underline{v}|| = 1$ esiste la derivata direzionale in
	$\underline{x}_0$ lungo la direzione $\underline{v}$. In particolare
	\[
		f_{\underline{v}}(\underline{x}_0) = \nabla f(\underline{x}_0) \cdot
		\underline{v}
	\]
\end{theorem}
\begin{proof} definizione di differenziabilità e di derivata direzionale
	\begin{itemize}
		\item Siccome $f$ è differenziabile in $\underline{x}_0$, allora per $h
				\to 0$
			\[
				f(\underline{x}_0 + h) = f(\underline{x}_0) + \nabla f
				(\underline{x}_0) \cdot \underline{h}
				+ o(||\underline{h}||)
			\]
		\item sia $\underline{h} = t \underline{v}$, allora per $t \to 0^+$
			\[
				f(\underline{x}_0 + t \underline{v}) = f(\underline{x}_0) +
				\nabla f (\underline{x}_0) \cdot t \underline{v}
				+ o(||t \underline{v}||)
			\]
		\item essendo $o(||t \underline{v}||) = o(|t| || \underline{v}||) =
				o(t)$, allora
			\[
				\frac{f(\underline{x}_0 + t \underline{v}) -
				f(\underline{x}_0)}{t} = \frac{\nabla f (\underline{x}_0) \cdot
				t \underline{v}}{t} + \frac{o(t)}{t}
			\]
		\item dunque per la definizione di derivata direzionale
			\[
				f_{\underline{v}}(\underline{x}_0)
				= \lim_{t \to 0} \frac{f(\underline{x}_0 + t \underline{v}) -
				f(\underline{x}_0)}{t}
				= \nabla f(\underline{x}_0) \cdot \underline{v}
			\]
	\end{itemize}
\end{proof}

\newpage
\begin{theorem} [Ortogonalità del gradiente agli insiemi di livello]
	Sia $A \subseteq \mathbb{R}^n$ aperto e $f: A \to \mathbb{R}$
	differenziabile in $A$, supponendo che l'insieme di livello $k \in
		\mathbb{R}$ di $f$ ($I_k = \{\underline{x} \in A: f(\underline{x}) = k
		\}$) sia il sostegno di una curva regolare $\underline{r}: I \subseteq
		\mathbb{R} \to A$, allora
	\[
		\nabla f(\underline{x}) \cdot r'(t) = 0 \quad \forall t \in I
	\]
\end{theorem}
\begin{proof} definizione e teorema di derivazione delle funzioni composte
	\begin{itemize}
		\item Considerando $F:I \to \mathbb{R}$, $F(t) = f(r(t))$
			si osserva che
			\[
				\{\underline{r}(t): t \in I\} = I_k = \{\underline{x} \in A:
				f(\underline{x}) = k \}
			\]
			dunque $F(t) = f(r(t)) = k$, da cui si deduce che $F'(t) = 0 \quad
				\forall t \in I$.
		\item per il teorema di derivazione delle funzioni composte 1-n-1 si
			ottiene che
			\[
				0 = F'(t) = \nabla f(\underline{r}(t)) \cdot \underline{r}'(t)
				\quad \forall t \in I
			\]
	\end{itemize}
\end{proof}

\newpage
\begin{theorem} [Criterio della matrice Hessiana]
	Siano $A \subseteq \mathbb{R}^n$ aperto e $f: A \to \mathbb{R}$ un
	aperto, $f \in C^2(A)$, $\underline{x}_0 \in A$ punto critico di $f$, $q :
		\mathbb{R}^n \to \mathbb{R}$ la forma quadratica indotta da
	$H_f(\underline{x}_0)$, allora:
	\begin{itemize}
		\item se $q$ è definita positiva $\Rightarrow$ $\underline{x}_0$ è un
			punto di minimo locale
		\item se $q$ è definita negativa $\Rightarrow$ $\underline{x}_0$ è un
			punto di massimo locale
		\item se $q$ è indefinita $\Rightarrow$ $\underline{x}_0$ è un punto di
			sella
	\end{itemize}
\end{theorem}
\begin{proof} definizione di forma quadratcia, formula di Taylor e definizione
	di limite
	\begin{itemize}
		\item $H_f(\underline{x}_0)$ è simmetrica per il teorema di Schwarz
		\item se $q$ è definita positiva $\Rightarrow$ $H_f(\underline{x}_0)$ ha
			autovalori $\lambda > 0 \in \mathbb{R}$
		\item quindi per la definizione di forma quadratica definita positiva
			\[
				H_f(\underline{x}_0) \underline{h} \cdot \underline{h} = q
				(\underline{h}) \geq \lambda_{min} ||\underline{h}||^2
				\quad \forall \underline{h} \in \mathbb{R}^n
			\]
		\item utilizzando la formula di Taylor al secondo ordine per
			$||\underline{h}|| \to 0$, dato che $x_0$ è un punto critico
			\begin{align*}
				f(\underline{x}_0 + \underline{h}) - f(\underline{x}_0) & =
				\nabla f(\underline{x}_0) \cdot \underline{h} + \frac{1}{2}
				q(\underline{h}) + o(||\underline{h}||^2)
				\\ & = \frac{1}{2} q(\underline{h})+ o(||\underline{h}||^2)
				\\ & \geq \lambda_{min} ||\underline{h}||^2 +
				o(||\underline{h}||^2)
			\end{align*}
		\item per la definizione di $o$ si ha che
			\[
				\lim_{||\underline{h}|| \to 0} \frac{o(||\underline{h}||^2)}
				{||\underline{h}||^2} = 0
			\]
		\item per la definizione di limite
			\[
				\exists \delta > 0 : \text{ se } ||\underline{h}|| < \delta
				\text{ e } \underline{h} \neq \underline{0} \Rightarrow
				\frac{o(||\underline{h}||^2)} {||\underline{h}||^2} <
				\frac{\lambda_{min}}{4}
			\]
			con $\lambda_{min} > 0$ perchè $q$ è definita positiva.\\
		\item In paricolare
			\[
				\exists \delta > 0 : \forall \underline{h} \in B_{\delta}(0)
				\Rightarrow o(||\underline{h}||^2) > -\frac{\lambda_{min}}{4}
				||\underline{h}||^2
			\]
		\item utilizzando la formula precedentemente trovata si deduce che
			$\forall h \in B_{\delta}(\underline{x}_0): \underline{x}_0 +
				\underline{h} \in A$
			\begin{align*}
				f(\underline{x}_0 + \underline{h}) - f(\underline{x}_0) & \geq
				\lambda_{min} ||\underline{h}||^2 + o(||\underline{h}||^2)
				\\ & \geq \lambda_{min} ||\underline{h}||^2 -
				\frac{\lambda_{min}}{4}
				\\ & \geq 0
			\end{align*}
			e quindi $\underline{x}_0 $ un punto di minimo locale
		\item analogamente si dimostra che se $q$ è definita negativa
			$\underline{x}_0$ è un punto di massimo locale
	\end{itemize}
\end{proof}

\newpage
\begin{theorem} [Cambiamenti di variabili in coordinate polari, cilindriche, e
	sferiche e il loro Jacobiano]
	Sia $\omega \subseteq \mathbb{R}^n$ un dominio regolare, $f: \omega
		\to \mathbb{R}$ una funzione continua, se $\underline{T} : U \to
		V$ è un cambio di varibili tra $U$ e $V$ con $\omega \subseteq V$
	\[
		\underline{T}(u_1, \dots, u_n) = (x_1 \dots, x_n),
	\]
	allora
	\[
		\int \dots \int_{\omega} f(x_1, \dots, x_n) dx_1 \dots dx_n
		= \int \dots \int_{\underline{T}^{-1}} f(\underline{T}(u_1, \dots, u_n))
		| \det(J_{\underline{T}}(u_1, \dots, u_n)) | du_1 \dots du_n
	\]
	\begin{itemize}
		\item Sia $\underline{T}_p : \mathbb{R}^+ \times [0,2\pi) \to
				\mathbb{R}^2$
			\[
				\underline{T}_p(\rho, \theta) =
				\begin{bmatrix}
					\rho \cos(\theta) \\
					\rho \sin(\theta) \\
				\end{bmatrix}
				=
				\begin{bmatrix}
					x \\
					y \\
				\end{bmatrix}
			\]
			il suo Jacobiano è
			\[
				det(J_{\underline{T}_p} (\rho, \theta)) = \rho
			\]
		\item Sia $\underline{T}_c : \mathbb{R}^+ \times
				[0,2\pi) \times \mathbb{R} \to \mathbb{R}^3$
			\[
				\underline{T}_c (\rho, \theta, z) =
				\begin{bmatrix}
					\rho \cos(\theta) \\
					\rho \sin(\theta) \\
					z                 \\
				\end{bmatrix}
				=
				\begin{bmatrix}
					x \\
					y \\
					z \\
				\end{bmatrix}
			\]
			il suo Jacobiano è
			\[
				det(J_{\underline{T}_c} (\rho, \theta, z)) = \rho
			\]
		\item Sia $\underline{T}_s : \mathbb{R}^+ \times
				[0,2\pi) \times [0,\pi] \to \mathbb{R}^3$
			\[
				\underline{T}_s(\rho, \varphi, \theta) =
				\begin{bmatrix}
					\rho \sin(\varphi) \cos(\theta) \\
					\rho \sin(\varphi) \sin(\theta) \\
					\rho \cos(\varphi)              \\
				\end{bmatrix}
				=
				\begin{bmatrix}
					x \\
					y \\
					z \\
				\end{bmatrix}
			\]
			il suo Jacobiano è
			\[
				det(J_{\underline{T}_s} (\rho, \theta, \varphi)) = \rho^2
				\sin(\varphi)
			\]
	\end{itemize}
\end{theorem}
\begin{proof} Dimostrazione riportata solo per il cambio di coordinate in
	coordinate sferiche: calcoli
	\begin{itemize}
		\item
			\[
				J_{\underline{T}_s} (\rho, \varphi, \theta) =
				\begin{bmatrix}
					\sin(\varphi) \cos(\theta) & \rho \cos(\varphi) \cos(\theta)
					                           & -\rho \sin(\varphi)
					\sin(\theta)
					\\ \sin(\varphi) \sin(\theta)& \rho \cos(\varphi)
					\sin(\theta)               & \rho \sin(\varphi) \cos(\theta)
					\\ \cos(\varphi) & -\rho \sin(\varphi)              & 0
				\end{bmatrix}
			\]
		\item
			\begin{align*}
				\det(J_{\underline{T}_s} (\rho, \varphi, \theta)) & =
				\cos(\varphi) \det \left(
				\begin{bmatrix}
						\rho \cos(\varphi) \cos(\theta) & -\rho \sin(\varphi)
						\sin(\theta)
						\\ \rho \cos(\varphi) \sin(\theta) & \rho \sin(\varphi)
						\cos(\theta)
					\end{bmatrix}
				\right)
				\\&\quad +
				\rho \sin(\varphi) \det \left(
				\begin{bmatrix}
						\sin(\varphi) \cos(\theta) & -\rho \sin(\varphi)
						\sin(\theta)
						\\ \sin(\varphi) \sin(\theta) & \rho \sin(\varphi)
						\cos(\theta)
					\end{bmatrix}
				\right)
				\\ &= \rho^2 \sin(\varphi) \cos^2(\varphi)
				(\cos^2(\theta) + \sin^2(\theta))
				\\ &\quad + \rho^2 \sin^3(\varphi) (\cos^2(\theta) +
				\sin^2(\theta))
				\\ &= \rho^2 \sin(\varphi) (\cos^2(\varphi) + \sin^2(\varphi))
				\\ &= \rho^2 \sin(\varphi)
			\end{align*}
	\end{itemize}
\end{proof}

\end{document}
