\documentclass[12pt, a4paper]{report}

\usepackage[margin=2cm]{geometry}
\usepackage{ragged2e}

\newtheorem{definition}{Definizione}

\title{Memoria}
\author{Gabr1313}
\date{\today}

\begin{document}
\justify
\sloppy
\maketitle
\tableofcontents

\chapter{Organizzazione dello spazio virtuale dei processi}
\section{VMA}
La memoria virtuale di un processo LINUX è suddivisa in un certo numero di
\textbf{aree di memoria virtuale} (\textbf{VMA}). Ogni VMA è costituita da un
numero intero di pagine virtuali consecutive.

Tipologie di VMA:
\begin{itemize}
	\item \textbf{Codice} (\textbf{C}): eseguibile e costanti.
	\item \textbf{Costanti per rilocazione dinamica} (\textbf{K}): parametri per
		il collegamento con librerie dinamiche.
	\item \textbf{Dati statici} (\textbf{S}): dati inizializzati allocati per
		tutta la durata del programma.
	\item \textbf{Dati dinamici} (\textbf{D}): heap (contiene anche i BSS). Il
		limite corrente è contenuto nella variabile del descrittore del
		processo \texttt{brk} (Program break).
	\item \textbf{Pile dei Thread} (\textbf{T}): contiene gli stack dei threads
	\item \textbf{Memory-Mapped files} (\textbf{M}): mappa di un file in aree di
		memoria virtuale. Ad esempio le \textbf{Librerie dinamiche} e la
		\textbf{Memoria condivisa}.
	\item \textbf{Memoria condivisa} (\textbf{P}): Upila che contiene le
		varibaili locali e i parametri delle funzioni.
\end{itemize}
Per C, K, S, D e P esiste un'area virtuale, mentre per M e per T possono
esistere zero o più aree virtuali. Le VMA C, K e S sono l'immagine dei
corrispondenti segmenti del file eseguibile (ELF). I \textbf{Block Started by
Symbol} (\textbf{BSS}) sono dati non inizializzati, che nell'eseguibile sono
rappresentati sinteticamente da un nome simbolico e lo spazio di memoria
occupato. Avendo un comportamento simile a quello di memoria allocata
dinamicamente, durante l'esecuzione sono allocati nello heap.

\subsection{Struttura dati per la VMA}
Ogni VMA è definita in una variabile di tipo \texttt{vm\_area\_struct}.
\begin{verbatim}
struct vm_area_struct {
    struct mm_struct *vm_mm; // puntatore al processo
    unsigned long vm_start;
    unsigned long vm_end;
    struct vm_area_struct *vm_prev; // linked list
    struct vm_area_struct *vm_next;
    unsigned long vm_flags;
    ...
    unsigned long vm_pgoff;
    struct file *vm_file;
    ...
    struct file * vm_file; // possibile puntatore al file
    unsigned long vm_pgoff; // possibile offset nel file
};
\end{verbatim}
Ogni bit di \texttt{vm\_flags} indica una caratteristica della VMA:
\texttt{VM\_READ}, \texttt{VM\_WRITE}, \texttt{VM\_EXEC}, \texttt{VM\_SHARED},
\texttt{VM\_GROWSDOWN}, \texttt{VM\_DENYWRITE}...

Una VMA può essere \textbf{anonima} o \textbf{mappata su un file} detto
\textbf{Backing Store}.

Il comando \texttt{cat /proc/<pid>/maps} mostra le VMA di un processo. I
permessi sono \texttt{r} (\texttt{VM\_READ}), \texttt{w} (\texttt{VM\_WRITE}),
\texttt{x} (\texttt{VM\_EXEC}), \texttt{p} (private: \texttt{VM\_SHARED}).

\subsection{Page fault}
Ogni volta che la MMU genera un interrupt di Page Fault su un inidirizzo
virtuale NPV, il kernel attiva la routine \textbf{Page Fault Handler}
(\textbf{PFH}).
\begin{verbatim}
if (NPV notin VMA || NPV hasn't permission) Segmentation Fault;
else if (NPV notin mem) load NPV in mem;
\end{verbatim}

\subsection{Regole generali e implementazione specifica}
\begin{itemize}
	\item \textbf{Regole generali}: garantite ufficialmente
	\item \textbf{Implementazione specifica}: comportamento non specificato
\end{itemize}

\subsection{Address Space Layout Randomization}
La \textbf{ASLR} è una tecnica di sicurezza che assegna casualmente l'indirizzo
delle funzioni di libreria e delle più importanti aree di memoria. Per
semplicità questa non viene considerata.

\section{Tabella delle Pagine e Aree Virtuali}
Il sistema operativo garantisce le seguenti regole:
\begin{itemize}
	\item In ogni istante esistono esclusivamente le pagine virtuali NPV
		appartenenti alle VMA del processo.
	\item La TP è sufficientemente grande
	\item NPV può essere mappata su una NPF (indicato da una flag sulla TP)
\end{itemize}
La tecnica \textbf{demand paging} consiste nel caricare in memoria le pagine in
base ai Page Fault generati.
\subsection{Riassunto}
\begin{itemize}
	\item viene creata la VMA di pila con n NPV, di cui ne vengono fisicamente
		allocate 1 o 2.
	\item l'accesso a NPV non mappate ne cause l'allocazione (Page Fault)
	\item tentativo di accesso a pagine subito sotto l'area esistente (area di
		growsdown) produce un tentativo di crescita della VMA
	\item tentativo di accesso a pagine non esistenti o diverse dall'area di
		growsdown produce un Segmentation Fault
\end{itemize}

\begin{verbatim}
if (NPV notin VMA || NPV hasn't permission) Segmentation Fault;
else if (NPV notin mem) {
    if (NPV is start_address of VMA with Growsdown == 1) add page NPV-1;
    load NPV in mem;
}
\end{verbatim}
\subsection{Servizio brk}
Con \textbf{brk} si intende \textbf{break}, ovvero il limite superiore di una
VMA.

La \texttt{malloc()} svolge 2 operazioni distinte:
\begin{itemize}
	\item crescita della VMA D (\texttt{brk()} o \texttt{sbrk()})
	\item allocazione fisica delle pagine (se necessario riempirle coi dati)
\end{itemize}
\texttt{int brk(void * end\_data\_segment)} assegna il valore
\texttt{end\_data\_segment} alla VMA D. Restituisce 0 in caso di successo, -1 in
caso di errore.
\texttt{void * sbrk(intptr\_t increment)} incrementa la VMA D di
\texttt{increment} e restituisce un puntatore alla posizione iniziale della
nuova area (program break position).

\section{Gestione della memoria virtuale nella creazione di processi}
\subsection{Fork}
Dato che il processo è una copia del padre, non viene allocata nuova memoria,
ma vengono solo aggiornate le informazione della tabella dei processi. Una pagina
viene duplicata solo nel momento di scrittura sulla pagina stessa: \textbf{Copy
on Write} (\textbf{COW}).
\begin{itemize}
	\item durante la \texttt{fork()}, i permessi di tutte le pagine vengono
		cambiati in sola lettura (R)
	\item quando si verifica un page fault, il PFH verifica se la pagina
		protetta R appartenga a una VMA scrivibile. In tal caso, se
		\texttt{ref\_count() > 1}, duplica le pagine. Successivamente cambia la
		protezione in scrittura (W).
\end{itemize}
La variabile \texttt{ref\_count} nel \texttt{page\_descriptor} indica il numero di
utilizzatori della pagina fisica.
\begin{verbatim}
if (NPV notin VMA) Segmentation Fault;
else if (NPV hasn't permission) {
    if (NPV is in VMA with COW) {
        if (ref_count() > 1) duplicate page;
        else change protection in W;
    }
    else Segmentation Fault;
}
else if (NPV notin mem) {
    if (NPV is start_address of VMA with Growsdown == 1) add page NPV-1;
    load NPV in mem;
}
\end{verbatim}
\subsection{Context Switch}
Un context switch causa un svuotamento (\textbf{flush}) della TLB (Table
Lookaside Buffer). Nel caso il descrittore della pagine fisica abbia
\texttt{dirty\_bit = 1} le pagine vengono copiate su disco.
\subsection{Exit}
\begin{itemize}
	\item eliminazione della struct della VMA
	\item eliminazione della TP del processo
	\item deallocazione delle NPV o decremento del \texttt{ref\_count}
	\item context switch (con flush del TLB)
\end{itemize}

\section{Creazione dei Thread e aree di tipo T}
I thread condividono la stessa struttura di aree virtuali e TP del processo
padre. L'unica area di memoria non condivisa è lo stack, la cui dimensione
massima è limitata a 8Mb (il flag \texttt{growsdown = 0}).

\section{Creazione di VMA e mmap}
La realizzazione di una VMA può essere invocata direttamente dal SO, o tramite
la syscall \texttt{mmap()}. Le VMA possono essere \textbf{mapped} su un file
(backing store) o \textbf{anonymous}, \textbf{shared} o \textbf{private} (la
combinazione anonymous-shared non è trattata).
\subsection{Mapped}
Nella struct \texttt{vm\_area\_struct} sono contenuti i campi \texttt{vm\_file}
e \texttt{vm\_pgoff}.

La funzione \texttt{mmap()} ha la seguente sintassi:
\begin{verbatim}
void * mmap(void * addr, size_t len, int prot, int flags, int fd, off_t off);
\end{verbatim}
\begin{itemize}
	\item \texttt{addr}: opzionale indirizzo di virtuale iniziale (Linux sceglie
		l'indirizzo disponibile più vicino)
	\item \texttt{len}: dimensione della VMA
	\item \texttt{prot}: permessi di accesso
	\item \texttt{flags}: \texttt{MAP\_SHARED}, \texttt{MAP\_PRIVATE} o
		\texttt{MAP\_ANONYMOUS}
	\item \texttt{fd}: file descriptor
	\item \texttt{off}: offset nel file (multiplo della dimensione della pagina)
\end{itemize}
In caso di successo \texttt{mapp()} restituisce un puntatore all'area allocata.

La funzione \texttt{munmap()} elimina la mappatura dell'intervallo virtuale
specificato:
\begin{verbatim}
int munmap(void * addr, size_t len);
\end{verbatim}

\subsection{Page Cache}
è un meccanismo che serve ad evitare la rilettura da disco di pagine già
caricate in memoria. è costituita da un insieme di pagine fisiche e strutture
dati ausiliarie. Le strutture dati ausiliarie contengono l'insieme dei
descrittori delle pagine fisiche presenti nella cache oltre che un meccanismo
\texttt{Page\_Cache\_Index} per la riesca di una pagina identificata da
\texttt{<indentificatore file, offset>}. Le pagine caricate nella Paga Cache non
vengono liberate nessun processo le sta utilizzando, ma solo quando la Page
Cache è piena.

Richiesta di accesso a una pagina virtuale mappata su file:
\begin{itemize}
	\item determinazione file e offset: il file è indicato nella VMA, e
		l'offset è la somma tra l'offset della VMA rispetto all'inizio del file
		e l'offset della pagina virtuale rispetto all'inizio della VMA
		(evidentemente nella VMA può essere caricata solo una porzione di un
		file).
	\item Se la pagina è nella Paga Cache, la pagina virtual e viene
		semplicemente mappata, altrimenti viene allocata una nuova pagina fisica
		nella paga cache.
\end{itemize}

\subsection{Scrittura sulle pagine}
\subsubsection{Scrittura su pagine condivise}
La pagina fisica viene scritta e il bit di dirty viene settato. I cambiamenti
sono immediatamente visibili da ogni processo.
\subsubsection{Scrittura su pagine private}
Il meccanismo di funzionamento è il COW. Le pagine modificate non mai ricopiate
su disco, in quanto appartengono al singolo processo.
\subsubsection{Scrittura su pagine anonime}
Lo stack e l'heap sono pagine anonime. Tutte le pagine anonime sono mappate
sulla \textbf{ZeroPage}, così che VMA trovi una pagina inizializzata a zero
senza richiedere una nuova allocazione. Il meccanismo di funzionamento è il COW.

\subsection{Mmap e fork}
Le aree di memoria create tramite mmap si trasmettono ai figli durante una fork
grazie al meccanismo di COW.

\subsection{Condivisione degli eseguibili}
Gli eseguibili sono VMA senza diritti di scrittura. Sono definite come aree
private (seppur ciò sia irrilevante). Anche il linker dinamico utilizza le VMA
private per condividere le pagine fisiche delle librerie.

\chapter{Gestione della memoria fisica}
Le allocazioni sono di 2 tipi:
\begin{itemize}
	\item \textbf{a grana grossa}: intere porzioni di memoria (\texttt{brk()}:
		unità di allocazione = pagina/gruppo di pagine)
	\item \textbf{a grana fine}: piccole strutture dati in porzioni gestite a
		grana grossa (\texttt{malloc()}
\end{itemize}
\section{Allocazione e deallocazione}
La RAM ha principalmente 3 usi:
\begin{itemize}
	\item sistema operativo
	\item processi
	\item pagine lette da disco (buffer/disk cache)
\end{itemize}
\subsection{Page reclaming}
Prima che avvenga una deallocazione, vengono salvate su disco le pagine con dirty
bit settato.
La deallocazione ha una gerarchia di priorità:
\begin{itemize}
	\item le pagine non più utilizzate dai processi vengono scaricate
	\item alcune pagine utilizzate dai processi vengono scaricate
	\item un processo viene ucciso: \textbf{OOMK} (Out Of Memory Killer)
\end{itemize}
Il comando \texttt{free} mostra lo stato della memoria.
Ovviamente avendo la memoria swap a disposizione, l'OOMK avviene più
tardivamente.
\section{Allocazione della memoria a grana grossa}
Linux cerca di allocare blocchi di pagine contigue e di frammentare la memoria il
meno possibile: il DMA è così più efficiente e la memoria risulta più compatta.

\begin{itemize}
	\item La dimensione di un blocco di pagine contigue è una potenza di 2,
		detta \textbf{ordine} (esiste una costante \texttt{MAX\_ORDER}).
	\item Per ogni ordine esiste una lista che collega tutti i blocchi.
	\item Le richieste di allocazione indicano l'ordine del blocco desiderato.
	\item se il blocco richiesto non è disponibile, viene diviso in 2 blocchi
		(\textbf{buddies}). Se necessario il processo viene reiterato.
	\item quando un blocco viene liberato, se il suo buddy è libero, i 2
		blocchi vengono riuniti.
\end{itemize}

\section{Deallocazione della memoria a grana grossa}
La memoria di un processo viene rilasciata appena il processo termina, mentre la
disk cache cresce indefinitivamente.

Quando la memoria scende sotto un livello critico \textbf{minFree}, interviene
la procedura \textbf{PFRA} (Page Frame Reclaiming Algorithm). PFRA necessita
esso stesso di memoria, per questo l'esistenza del minFree. Per ridurre il
numero di attivazioni di PFRA, si cercano di scaricare \textbf{maxFree} pagine.

La scelta delle pagine da liberare si basa sui principi di LRU.

\section{Determinazione delle pagine da liberare}
\subsection{Quante}
\begin{itemize}
	\item \textbf{freePages}: numero di pagine libere.
	\item \textbf{requiredPages}: numero di pagine richieste da un processo o
		dal SO.
	\item \textbf{maxFree}: numero di pagine che il PFRA tenta di liberare.
	\item \textbf{minFree}: livello critico di memoria libera.
\end{itemize}
Il PFRA viene invocato se:
\begin{itemize}
	\item \texttt{freePages - requiredPages < minFree}
	\item \texttt{kswapd} (kernel swap daemon): funzione invocata periodicamente
		se \texttt{freePages < maxFree}
\end{itemize}
Pfra cerca quindi di liberare \texttt{toFree = maxFree - freePages +
requiredPages} pagine.

In caso di carico leggero \texttt{kswapd} causa rare attivazione dirette di
PFRA, viceversa con uso intensivo della memoria.

Nel caso in cui il PFRA non riesce a liberare pagine e la memoria residua è
insufficiente, viene invocato l'OMMK. L'OMMK chiama la funzione
\texttt{select\_bad\_process()} che considera i seguenti criteri: il processo
abbia molte pagine occupate, abbia svolto poco lavoro, abbia bassa priorità,
non abbia privilegi di root, non gestisca componenti HW (rimangono altrimenti in
stato incosciente) e non sia un kernel thread.

Potrebbe verificarsi uno o stato di \textbf{Thrashing}: i processi
continuano a richiedere pagine che vengono deallocate senza fare progredire con
l'esecuzione.
\subsection{Quali}
Le pagine si dividono in 4 categorie:
\begin{itemize}
	\item \textbf{Non scaricabili}: pagine di sistema (statiche o dinamiche) e
		pagine della sPila dei processi.
	\item \textbf{Richiedono una Swap} per essere scaricate: pagine dati, uPila
		e Heap.
	\item \textbf{Buffer}: pagine mappate su file
	\item \textbf{Mappate su file eseguibile}: non necessitano di riscrittura.
\end{itemize}
Esistono 2 liste globali, le \textbf{LRU list}, che contengono tutte le pagine
appartenenti ai processi:
\begin{itemize}
	\item \textbf{Active list}: non possono essere scaricate
	\item \textbf{Inactive list}
\end{itemize}
L'ordine delle liste è un'approssimazione di LRU: l'x64 non tiene traccia del
numero di accessi alla memoria, ma utilizza solamente un bit (\textbf{A}) di
accesso presente nel TLB (translation lookaside buffer) gestito dall'HW. Questo
bit viene settato ogni volta che la memoria viene acceduta, e riportato a 0
periodicamente dal kernel.

Oltre a ciò, ad ogni pagina viene associata una flag \textbf{ref} (referenced).
Periodicamente \texttt{kswapd} esegue una funzione di controllo delle liste:
\begin{itemize}
	\item Scansione della coda della active list:
        \begin{verbatim}
    if (A) {
        A = 0;
        if (ref) move Page to the head of active list;
        else ref = 1;
    } else {
        if (ref) ref = 0;
        else move Page to the head of inactive list;
    }
        \end{verbatim}
	\item Scansione della testa della inactive list:
        \begin{verbatim}
    if (A) {
        A = 0;
        if (ref) {
            ref = 0;
            move Page to the head of active list;
        } else ref = 1;
    } else {
        if (ref) ref = 0;
        else move Page to the tail of inactive list;
    }
        \end{verbatim}
	\item Caricamento di nuove pagine
		\begin{itemize}
			\item se richieste da un processo vengono poste in testa alla active
				con \texttt{ref = 1}
			\item le pagine di un nuovo processo possiedono lo stesso
				\texttt{ref} del padre e poste in testa alla stessa lista del
				padre.
		\end{itemize}
	\item Eliminazione di pagine: swapped out o exit dal processo.
\end{itemize}

\subsection{Scaricamento delle pagine}
\begin{itemize}
	\item Vengono prima scaricate le pagine della Page Cache non più utilizzate
		in ordine di NPF.
	\item Vengono poi scaricate le pagine fisiche della inactive list (che
		contiene NPV). Ovviamente una NPF viene scaricata solo se si incontrano
		tutte le NPV che a lei fanno riferimento.
\end{itemize}

\section{Swapping}
Il meccanismo di swapping necessita almeno una \textbf{Swap Area} (risiedente su
file o partizione). Questa contiene dei \textbf{Page slot} che hanno dimensione
di 1 pagina, identificati da $<$SwapArea, PageSlot$>$ (\textbf{SWPN}). A ogni
Swap Area è associato un \textbf{descrittore}, contente tra le altre cose un
\texttt{swap\_map\_counter} per ogni slot, che tiene traccia del numero di PTE
(page table entries) riferiti all'area swappata.
\subsection{Swap Out}
Quando una pagina viene liberata, se il bit di dirty è 0, viene eliminata,
altrimenti viene salvata. Se la pagina è mappata su file, viene salvata in tale
file; se invece è anonima, viene salvata nella Swap Area.
\begin{itemize}
	\item Allocazione di un SWPN
	\item Copia di NPF in SWPN
	\item aggiramento di \texttt{swap\_map\_counter}
	\item ogni PTE che condivideva la pagina fisica viene aggiornata:
		NPF viene sostituito da SWPN, e il bit di presenza viene
		resettato.
\end{itemize}
Per determinare se una pagina fisica condivisa sia dirty, oltre a controllare il
bit di dirty della pagina stessa, bisogna controllare nel TLB se le pagine
virtuali a lei associata siano dirty. % TODO: WHY?
\subsection{Accesso alla swap}
\begin{itemize}
	\item page fault
	\item gestore del page fault chiama \texttt{swap\_in}
	\item allocazione pagina fisica in memoria
	\item Copia di SWPN (indicato in PTE) in NPF
	\item ogni PTE che condivideva la pagina fisica viene aggiornata: SWPN viene
		sostituito da NPF, e il bit di presenza viene settato.
\end{itemize}
\subsection{Swap In}
Se avviene la Swap In di una pagina, e successivamente una Swap Out senza che ci
siano state modifiche al contenuto della pagina stessa, Linux cerca di
riutilizzare lo stesso SWPN. Utilizza quindi una \textbf{Swap Cache}, una
struttura dati simile alla Page Cache. Una pagina appartiene alla Swap Cache
viene registrata nello \textbf{Swap Cache Index} se è presente sia in Memoria,
che nella Swap Area.
Le pagine presenti in Swap sono solo anonime (e quindi private nella nostra
semplificazione). Nel momento in cui avviene una Swap In:
\begin{itemize}
	\item La pagine viene copiata in memoria fisica, marcata in sola lettura
		come pagina condivisa (dalla swap e dal processo).
	\item Nella Swap Cache Index viene inserito un descrittore contenente i
		riferimenti sia alla Pagina Fisica che al Page Slot.
\end{itemize}
Se poi la pagina viene scritta, si verifica un Page Fault
\begin{itemize}
	\item La pagina diventa privata (non appartiene più alla Swap Area)
	\item La sua protezione viene posta a W
	\item \texttt{swap\_map\_counter--}
	\item Se \texttt{swap\_map\_counter == 0} il Page Slot viene liberato nella
		Swap Area.
\end{itemize}
Successivamente a una Swap In
\begin{itemize}
	\item la NPV viene inserita in testa alla lista active con \texttt{ref = 1}
	\item altre NPV condivise all'interno della stessa pagina vengono poste in
		coda alla lista active con \texttt{ref = 0} \textit{(penso ci sia un
		errore: le NPV in swap dovrebbero essere solo private nella nostra
		semplificazione)}
\end{itemize}

\chapter{Tabella della pagine}
\section{Struttura della memoria virtuale del SO}
Gli indirizzi virtuali si estendono da \texttt{FFFF 8000 0000 0000} a
\texttt{FFFF FFFF FFFF FFFF} per un totale di $2^{47}$B = 128TB.
\begin{itemize}
	\item Codice e dati statici: 0.5GB
	\item Moduli a caricamento dinamici: 1.5GB
	\item Mappatura della memoria fisica: 64TB (50\%)
	\item Strutture dinamiche
	\item Mappature memoria virtuale
	\item altro
\end{itemize}
\subsection{Page offset}
Il sistema operativo opera su indirizzi virtuali (come gli altri processi), ma
deve anche essere in grado di accedere agli indirizzi fisici, ad esempio per
gestire la tabella della pagine (una struttura dati HW molto grande). Questo
problema è in parte risolto mappando 1:1 la memoria fisica. L'indirizzo
iniziale di questa area virtuale è chiamato \texttt{PAGE\_OFFSET}.
Le aree di sPila dei processi sono allocate dal SO in un indirizzo fisico e
mappate sul corrispondente indirizzo virtuale per essere utilizzate.
\section{Paginazione in x64}
\subsection{Tabella della pagine}
La MMU (Memory Management Unit) è l'unità di gestione della memoria a livello
HW: può essere un componente della CPU o esterno.

L'indirizzo di una pagina virtuale (di 4KB) è suddiviso in 12 bit di offset e
36 bit di NPV (per un totale di $2^{36}$ pagine virtuali). Ovviamente le pagine
virtuali non sono tutte allocate sulla TP: perciò la TP è organizzata come un
albero su 4 livelli:
\begin{itemize}
	\item i 36 bit di NPV sono suddivisi in 4 parti da 9 bit
	\item ogni gruppo da 9 bit rappresenta l'offset nella \textbf{directory}:
		una tabella le cui right sono chiamate \textbf{PTE} (Page Table Entry)
	\item La dimensione di ogni directory è 4KB: ogni PTE occupa 64 bit.
	\item L'indirizzo della directory principale è contenuto nel registro
		\texttt{CR3} (Control Register 3) della CPU.
\end{itemize}
\subsection{Page Walk}
Per tradurre un NPV in NPF:
\begin{itemize}
	\item accedi a \texttt{CR3}
	\item leggi la PTE indicata dall'offset di primo livello di NPV
	\item leggi l'indirizzo di base della directory di livello inferiore
	\item ripeti la procudura per ogni livello
\end{itemize}
Ad ogni livello, la PTE non contiene solo l'indirizzo del livello inferiore, ma
anche dei \textbf{flags}, memorizzati nei 12 bit bassi non utilizzati (essendo
l'offset della TP 0).
\subsection{Translation Lookaside Buffer}
Essendo che la Page Walk richiede 5 accessi a memoria, l'architettura x64
possiede un TLB in cui conservare le corrispondenze tra NPV e NPF più
recenti. Il TLB è gestito dall'MMU in maniera trasparente al software.
Se il registro CR3 viene modificato la MMU invalida tutte le PTE del TLB,
escluse quelle globali (flag G).
\subsection{Struttura della TP}
Le 4 directory sono:
\begin{itemize}
	\item \textbf{PGD}: Page Global Directory
	\item \textbf{PUD}: Page Upper Directory
	\item \textbf{PMD}: Page Middle Directory
	\item \textbf{PT}: Page Table
\end{itemize}
\subsection{Avviamento del SO}
Deve esistere un meccanismo che all'avvio del sistema permetta di caricare la
TP. Durante l'avvio le funzioni di caricamento accedono direttamente alle
memoria fisica (paginazione non attiva).

La tabella delle pagine è infatti memorizzata sulla memoria fisica anch'essa.
La tabella della pagine, quindi, rimappa anche se stessa.
\subsection{TP del Kernel}
Non esistendo una TP del SO, il SO viene mappato nella metà superiore dalla TP
di ogni processo. Le metà superiori di ogni processo puntano alla stessa
sottodirectory, relativa al SO (memorizzata fisicamente una volta sola).

\section{Dimensione della TP}
\begin{itemize}
	\item una PTE di PMD mappa 512 x 4KB = 2MB
	\item una PTE di PUD mappa 512 x 2MB = 1GB
	\item una PTE di PGD mappa 512 x 1GB = 512GB
\end{itemize}
L'occupazione percentuale di memoria nella TP rispetto alla memoria fisica
diminuisce col numero della pagine fisiche utilizzate tendendo a 1/512. Con
programmi molto piccoli, tuttavia può essere maggiore di 1.

\section{TP VMA e thread}
Gli stack crescono verso indirizzi bassi, mentre le VMA verso indirizzi alti.
La pagina finale di uno stack è la prima pagina di una VMA, e viceversa.

I thread condividono la stessa TP del padre, differendo solo per quanto riguardo
lo stack. Ogni thread ha bisogno di 2048 + 1 pagina per lo stack, e quindi 4 PT
+ 1 pagina.

\section{Gestione della TP da software}
\begin{itemize}
	\item \texttt{mk\_pte()} covert in una PTE un numero di pagina fisica e
		una serie di flag. 
	\item \texttt{pdg\_alloc()}, \texttt{pud\_alloc()}, \texttt{pmd\_alloc()},
		\texttt{pte\_alloc()} allocano e inizializzano intere pagine della TP.
	\item \texttt{pgd\_free()}, \texttt{pud\_free()}, \texttt{pmd\_free()},
		\texttt{pte\_free()} liberano intere pagine della TP.
	\item \texttt{set\_pdg()}, \texttt{set\_pud()}, \texttt{set\_pmd()},
		\texttt{set\_pte()} modificano una PTE.
\end{itemize}
Quando viene richiesto l'accesso a una NPV non ancora mappata sulla TP, il So
deve inserire una nuova PTE nella directory PT (se questa già esiste), oppure
allocare una nuova PT, e inserire la PTE nella directory PMD. Se questa non
esiste la procedura viene ripetuta.

\section{Approfondimenti}
\subsection{TLB}
La TLB è gestita in modo trasparente dall'HW. La TLB flush viene eseguita ogni
volta che viene modificato il valore di CR3 (ad esempio durante un context
switch). Le pagine di sistema più utilizzate sono marcate come globali, e non
vengono perciò eliminate.

\subsection{Cache}
Anche la cache è gestita in modo trasparente dall'HW.

\subsection{Context switch}
La funzione \texttt{schedule()}, prima di invocare la macro
\texttt{switch\_to(prev, next)}, invoca la funzione \texttt{switch\_mm(oldmm,
mm, next)}. Questa funzione invoca la macro \texttt{load\_cr3(next->pgd)} che
carica in CR3 il nuovo valore.

\end{document}
