\documentclass[12pt, a4paper]{report}

\usepackage[margin=2cm]{geometry}
\usepackage{ragged2e}
\usepackage{amssymb}
\usepackage{amsmath}
\usepackage{hyperref}


\newtheorem{definition}{Definizione}

\title{Memoria}
\author{Gabr1313}
\date{\today}

\begin{document}
\justify
\sloppy
\maketitle
\tableofcontents

\chapter{Il File System}
Fornisce un livello di astrazione (\textbf{Modello d'utente}) per consentire
l'accesso alle periferiche (file speciali).
\section{Drivers}
Le tipologie di periferiche sono molteplici, e queste evolvono nel tempo. Linux
permette quindi di aggiungere dei moduli software al SO: i \textbf{device
drivers}.

Questi vengono caricati solo quando necessari. Il meccanismo di inserimento nel
sistema di nuovo software è chiamato \texttt{kernel\_modules}. A un livello
astratto i driver devono avere un comportamento omogeneo.

I driver si dividono in due categorie:
\begin{itemize}
	\item \textbf{Driver a carattere}: le operazioni sono eseguite,
		ordinatamente, direttamente dai programmi che le richiedono.
	\item \textbf{Driver a blocchi}: il sistema pone le sue richieste in una
		coda, gestita da un FS. Il driver la può modificare così da ottimizzare
		le prestazioni. \end{itemize}

\section{Virtual File System e VFS}
Linux è in grado di gestire una moltitudine di file system (compatibilità,
ottimizzazioni \dots). Il \textbf{Virtual File System Switch} è lo strato che
premette ai programmi applicativi di essere indipendenti dall'implementazione
del file system: ridirigono le richieste di servizi alle corrette routine.
L'interfaccia unica è chiamata \textbf{modello del VFS}.

Più FS possono coesistere nello stesso sistema, con il vincolo che ogni
partizione (\textbf{Device}) sia essere gestita da un solo FS. Ogni Device è
costituito da un insieme di blocchi identificati da un \textbf{LBA} (Logical
Block Address).

\section{Ottimizzazioni con la memoria}
L'accesso a periferiche come un disco può richiedere diversi millisecondi:
perciò Linux tenta di mantenere in memoria i dati letti da disco il più lungo
possibile. Il VFS deve quindi collaborare strettamente col gestore della
memoria. Quando il VFS o un FS ha bisogno di accedere a un Device, la richiesta
passa per la \textbf{Page Cache} che si occupa dell'eventuale trasferimento in
memoria del blocco.

\chapter{Modello d'utente}
\section{Modello per l'accesso al singolo file}
L'accesso a un file può avvenire secondo 2 modalità:
\begin{itemize}
	\item mappatura di una VMA sul file (\texttt{mmap()})
	\item system calls, le cui API permettono di creare, cancellare, leggere o
		scrivere su file, oppure di creare o cancellare i direttori.
\end{itemize}
\subsection{Modello di un file}
Un file è costituito da una sequenza di byte memorizzati su disco. Un programma
non lavora mai direttamente su tale contenuto, ma su una sua copia in memoria.

Esiste un'\textbf{indicatore della posizione corrente} che punta al byte in cui
avviene l'operazione di lettura o scrittura.

Al momento dell'apertura un file (identificato dal nome) restituirci un intero
non negativo: il \textbf{descrittore del file} (un puntatore).

\subsubsection{Funzioni della libreria di Linux}
\begin{itemize}
	\item \texttt{int open(char *file\_name, int type, int permission)}:
		\texttt{pathname} è il nome completo; \texttt{type} indica scrittura,
		lettura o entrambi.
	\item \texttt{int creat(char *file\_name, int permission)}
	\item \texttt{int close(int fd)}: elimina il legame tra descrittore e file.
		Non è garantita la scrittura su disco.
	\item \texttt{int fsync(int fd)}: forza la scrittura su disco.
	\item \texttt{int read(int fd, void *buffer, int size)}: lettura
	\item \texttt{int write(int fd, void *buffer, int size)}: scrittura
	\item \texttt{long lseek(int fd, long offset, int whence)}: modifica la
		posizione corrente (permette di non operare in maniera sequenziale su un
		file) e restituisce il valore della nuova posizione. \texttt{whence} può
		essere 0, 1 o 2: inizio del file, posizione corrente, fine del file.
\end{itemize}

\section{Organizzazione complessiva dei file}

\subsection{Directory e filename}
I file e le partizioni sono organizzati in una struttura ad albero i cui nodi
sono detti \textbf{directory} (folder\dots). Le directory sono dei file (di
\textbf{tipo directory}) che contengono i nomi e le informazioni di accesso ad
altri file (sia di \textbf{tipo normale} che di tipo directory). Per scrivere o
leggere su questi file si utilizzano servizi speciali.

Il \textbf{Pathname} o nome completo è costituito dal nome delle directory
separate da uno "/". La \textbf{root} è indicata da "/".

\begin{itemize}
	\item \texttt{link()}: permette di creare un nuovo nome per un file già
		esistente
	\item \texttt{unlink(char *file\_name)}: rimuove il riferimento a un file.
		In Linux non esiste un servizio per la cancellazione di un file, ma
		semplicemente se un file rimane senza riferimenti, quello spazio diventa
		libero.
	\item \texttt{mkdir()}: è l'equivalente di \texttt{creat()} per le
		directory.
\end{itemize}

\subsection{Periferiche e file speciali}
Tutte le periferiche sono considerate come \textbf{tipo speciale}. I file
speciali sono trattati come normali file, con la differenza che alcune delle
operazioni perdono di significato. Spesso i file speciali risiedono nella
cartella \texttt{/dev}.

In questa cartella sono presenti anche i terminali virtuali (creati dalla gui),
di solito indicati con \texttt{/dev/pts/`n`}. A differenza dei dispositivi fisici,
la creazione di terminali virtuali avviene su comando. Per conoscere il nome del
file speciale associato ad un terminale si usa il comando \texttt{tty}.

L'interprete dei comandi (shell) fa in modo che un programma in esecuzione abbia
i descrittori 0, 1 e 2: \textbf{stdin}, \textbf{stdout} e \textbf{stderr} (file
speciali). Questi descrittori possono essere rediretti su file normali: quando
ad esempio si usa la funzione \texttt{close(0)}, la stdin viene rediretta sul
file successivamente aperto.

La funzione \texttt{dup()} duplica un descrittore, seppur l'indicatore della
posizione rimane univoco.

\subsubsection{Creazione di file speciali}
Ogni periferica è identificata da una coppia di numeri: \textbf{major} e
\textbf{minor}. Il primo identifica il driver, il secondo il dispositivo.
L'equivalente di \texttt{creat()} per i file speciali è \texttt{mknod(pathname,
type, major, minor)}, la quale può essere utilizzata solo dall'amministratore
del sistema (type = character / block).

\section{I Device}
A differenza di Windows, in cui i device sono identificati da una lettera (C:,
D:) che india la radice dell'albero, in Linux i device sono rappresentati da
nodi detti \textbf{mount point}. Il sottoalbero ne descrive la struttura
interna.

Per inserire un nuovo device nel sistema si deve:
\begin{itemize}
	\item associare un FS al device (con \texttt{mkfs -t type device})
	\item montare il volume in un mount point (con \texttt{mount device
		mount\_point})
\end{itemize}
A questo punto i file possono essere acceduti sia dal mount point che dal
device.

\chapter{Il Modello del VFS}
Le informazioni contenute nel VFS sono di 2 tipi: quelle statiche relative al
contenuto dei file e quelle dinamiche associate ai file aperti.

Le strutture dati su cui si base il VFS sono:
\begin{itemize}
	\item \texttt{struct dentry} (directory entry): ogni istanza rappresenta una
		directory nel VFS.
	\item \texttt{struct inode}: ogni istanza rappresenta un file fisicamente
		presente nei device.
	\item \texttt{struct file}: ogni istanza rappresenta un file aperto.
\end{itemize}
Queste strutture dati sono accedute tramite puntatori contenuti in:
\begin{itemize}
	\item \textbf{struttura delle directory}
	\item \textbf{struttura di accesso dai processi ai file aperti}
\end{itemize}

\section{Struttura delle directory}
Questa struttura dati ad albero è costituita da \texttt{struct dentry}.
\begin{verbatim}
struct dentry {
    struct inode *d_inode;
    struct dentry *d_parent;
    struct qstr d_name;
    struct list_head d_subdirs;
    struct list_head d_child;
    ...
}
\end{verbatim}
\begin{itemize}
	\item \texttt{d\_inode} è un puntatore alla struttura che rappresenta il
		file fisico.
	\item \texttt{d\_name} è il nome del file o della directory.
	\item \texttt{d\_subdirs} è un puntatore al primo dei figli.
	\item \texttt{d\_child} è un puntatore al prossimo fratello.
\end{itemize}

\section{Struttura di accesso dai processi ai file aperti}
Nel descrittore dei processi sono presenti 2 puntatori rilevanti ai fini
dell'accesso ai file:
\begin{verbatim}
struct task_struct {
    struct fs_struct *fs;
    struct files_struct *files;
    ...
}
\end{verbatim}
\begin{itemize}
	\item \texttt{fs} è un puntatore alla lista dei file system utilizzati dal
		processo.
\end{itemize}
\begin{verbatim}
struct files_struct {
    int next_fd;
    struct file **fd_array[NR_OPEN_DEFAULT];
    ...
}
\end{verbatim}
\begin{itemize}
	\item Ogni elemento di \texttt{fd\_array} è un puntatore a \texttt{struct
		file}, una struttura dati utilizzata per rappresentare i file aperti.
	\item \texttt{fd}, descrittore del file F, è l'indice di \texttt{fd\_array}.
\end{itemize}

\begin{verbatim}
struct file {
    struct list_head f_list;
    struct dentry *f_dentry;
    loff_t f_pos;
    int f_count;
    ...
}
\end{verbatim}
\begin{itemize}
	\item \texttt{f\_list} serve a collegar tutti i file aperti in una lista.
	\item \texttt{f\_dentry} è un puntatore alla directory entry utilizzata per
		aprire il file.
	\item \texttt{f\_pos} è l'indicatore della posizione corrente.
	\item \texttt{f\_count} è il numero di processi che hanno aperto il file.
\end{itemize}
Il percorso di un file aperto è quindi:
\texttt{file\_descriptor->files.fd\_array[fd]->f\_dentry->d\_inode}.

Quando un nuovo file viene aperto, viene creata una nuova istanda di
\texttt{struct file} e viene inserita nella prima posizione libera della tabella
dei file aperti. Se non sono già presenti in memoria vengono create anche le
struct \texttt{dentry} e \texttt{inode} (eliminate quando \texttt{f\_count}
diventa 0).

L'operazione di \texttt{fork()} duplica la tabella dei file aperti usando la
funzione \texttt{dup()}, quindi il riferimento a \texttt{struct
file} di tutti i file aperti è lo stesso.

Se lo stesso file fisico viene aperto da un'altra posizione (link), invece,
viene creata una nuova istanza di \texttt{struct file}, il quale avrà una
\texttt{f\_pos} differente, che punta a una \texttt{f\_dentry}, la quale però
punta  allo stesso \texttt{inode}.

\section{Struct inode}
La corrispondenza tra \texttt{struct inode} e file fisico è 1 a 1.
\begin{verbatim}
struct inode {
	loff_t i_size;
    struct list_head i_dentry;
    struct super_block *i_sb;
    struct inode_operations *i_op;
    struct file_operations *i_fop;
    struct address_space *i_mapping;
    ...
}
\end{verbatim}
\begin{itemize}
	\item \texttt{i\_size} è la dimensione del file.
	\item \texttt{i\_dentry} è l'inizio della lista dei dentry del file (un file
		può avere più nomi: link).
	\item \texttt{i\_sb} è un puntatore al superblocco che lo gestisce: una
		struttura dati che contiene le caratteristiche del FS.
	\item \texttt{i\_op} e \texttt{i\_fop} contengono i puntatori alle funzioni
		dello specifico FS (non risiedono nel blocco per rendere le operazioni
		più immediate).
		\begin{itemize}
			\item \texttt{i\_op} contiene le operazioni per la gestione
				complessiva dei file e delle directory:\\
				\texttt{create}, \texttt{mknod}, \texttt{link}, \texttt{unlink},
				\texttt{mkdir}, \texttt{rmdir} \dots
			\item \texttt{i\_fop} contiene le operazioni di accesso ai file:\\
				\texttt{read}, \texttt{write}, \texttt{open}, \texttt{close},
				\texttt{llseek} \dots
		\end{itemize}
	\item \texttt{i\_mapping} contiene informazioni per il trasferimento dei
		dati da device a memoria.
\end{itemize}

\section{Accesso ai dati}
\texttt{read()} e \texttt{write()} chiamano delle system calls. Il disco è
suddiviso in LBA da 1024 byte e la page cache in pagine da 4096 byte. La lettura
di un file è basata sulla pagina: è l'unità minima trasferita dal SO. Una
pagina è quindi costituita da 4 blocchi, ognuno identificato dal \textbf{File
Block Address} (\textbf{FBA}).
\begin{itemize}
	\item determina la pagina del file alla quale i byte appartengono:
		\texttt{OFFSET = POS \% 4096}.
	\item verifica se la pagina è contenuta nella page cache.
	\item altrimenti alloca una nuova pagina nella page cache e carica 4 LBA:
		\texttt{FP = POS / 4096}.
	\item copia i dati richiesti all'indirizzo richiesto dalla system call.
\end{itemize}
Blocchi di FBA consecutivi possono essere allocati in LBA non consecutivi
(dipende dalla disponibilità di risorse e dal FS utilizzato).
\subsection{Page cache}
La page cache contiene pagine fisiche, strutture dati e funzioni. Le strutture
dati contengono i descrittori delle pagine fisiche, i quali a loro volta
contengono l'identificatore del file, l'offset, il ref\_count \dots. Una pagina
è identificata dalla coppia \texttt{<file, offset>}.

Il campo \texttt{i\_mapping} di \texttt{struct inode} punta a
\begin{verbatim}
	struct address_space {
	struct inode host;
	struct radix_tree_root page_tree;
	struct ... a_ops;
	...
	}
\end{verbatim}
Questa struttura serve a determinare se una pagina di un file è presente nella
page cache.
\begin{itemize}
	\item \texttt{page\_tree} è una struttura dati che punta a tutte le pagine
		nella Page Cache relative al file. Se la pagina non viene trovata, si
		procede a caricarla.
	\item \texttt{a\_ops} contiene operazioni specifiche del FS per accedere
		alle pagine:\\
		\texttt{readpage}, \texttt{writepage}, \dots \\
		Queste solitamente chiamano funzioni del device driver.
\end{itemize}
Questo processo supporta anche la mappatura di una VMA su un file.
\subsubsection{Altro}
\begin{itemize}
	\item Esistono anche la \texttt{inode\_cache} e la \texttt{dentry\_cache},
		separate dalla page cache.
	\item esiste la possibilità di caricare sulla page cache un numero di
		blocchi che non è multiplo della dimensione della pagina.
\end{itemize}

\chapter{Gli extended FS ext2, ext3 e ext4}
\section{Introduzione}
I FS default di windows sono \textbf{ext2} e \textbf{ext3} e \textbf{ext4} (le
ultime sono delle versioni aggiornate e compatibili con la prima).

Il journaling è una tecnica che evita la creazione di file corrotti in caso di
chiusura anomala.
\section{Il FS ext2}
Ogni device è suddiviso in \textbf{Block Groups}

\subsection{ext2 senza Block Groups}
Le principali strutture dati utilizzate sono:
\begin{itemize}
	\item \textbf{superblock}: contiene le informazioni globali e viene
		eventualmente utilizzato durante il boot.
	\item \textbf{tabella degli inode}: array di inode
	\item \textbf{inode}: informazioni relative al singolo file
	\item \textbf{directories}
	\item \textbf{blocchi dati}: il contenuto effettivo dei file
\end{itemize}
Il superblock ha un indirizzo prefissato: da esso sono raggiungibili la tabella
degli inode e la root directory. La root directory può coincidere con la root
del sistema o con un mount point.

A ogni istante un blocco dati appartiene a un file oppure alla \textbf{lista
libera}.

\subsubsection{Contenuto di un inode}
Contiene le informazioni generali relative a un file. Nei cataloghi i file sono
individuati dalla coppia \texttt{<nome, numero di inode>}. Le informazioni più
importanti sono:
\begin{itemize}
	\item \textbf{tipo}: normale, directory o speciale
	\item \textbf{numero di riferimenti} (link) al file
	\item \textbf{dimensione}
	\item \textbf{puntatori ai blocchi dati} a eccetto dei file speciali.
\end{itemize}

\subsubsection{Accesso ai Blocchi dati}
La dimensione degli FBA (1KB - 64KB) deve essere minore o uguale a quelli di una
pagina (4KB su x64).

In ext2 l'accesso ai blocchi è basato su 15 puntatori. I primi 12 puntano
direttamente ai blocchi dati, gli ultimi 3 sono puntatori indiretti (uno
singolo, uno doppio e uno triplo). Questa struttura serve a rendere più
efficiente l'accesso ai file di piccole dimensioni.

Se $b$ è la dimensione di un blocco, allora la massima dimensione del file è
\[
	b((b/4)^3 + (b/4)^2 + b/4 + 12)
\]
$b/4$ è la dimensione di un blocco diviso per la dimensione di un puntatore.

\subsection{ext2 con Block Groups}
Un Block Group è un insieme di LBA contigui. L'accesso consecutivo a LBA
consecutivi è più veloce di un accesso casuale: è compito del driver disporre
vicini blocchi con informazioni correlate.

La dimensione di un Block Group è \texttt{8 * block\_size}: in un blocco è
memorizzata una bitmap che indica l'utilizzo o meno degli altri blocchi.

Il superblocco del volume è unico, ma può essere replicato in più block
group.

La tabella degli inode è logicamente unica, suddivisa in parti uguali nei
diversi block group. Nel superblocco è memorizzato il parametro
\texttt{inodes\_per\_group}.
\begin{verbatim}
    bg = (inode_num - 1) / inodes_per_group	
offset = ((inode_num - 1) % inodes_per_group) * inode_size
\end{verbatim}

In generale, le strutture dati contenute in un Block Group sono:
\begin{itemize}
	\item \textbf{Superblock}: contiene le informazioni globali (opzionale)
	\item \textbf{Block group descriptor table}
	\item \textbf{Block usage bitmap}
	\item \textbf{Inode usage bitmap}
	\item \textbf{Porzione della tabella degli inode}: contenente  soli
		\texttt{inodes\_per\_group} inode
	\item \textbf{blocchi dati}: il contenuto effettivo dei file
\end{itemize}

\section{Extent in ext4}
Un \textbf{extent} è un insieme di blocchi contigui sia all'interno del file
che sul dispositivo fisico. Un extent è rappresentato da \texttt{<FBA iniziale,
LBA iniziale, dimensione>}: non sono più necessari i puntatori ai singoli
blocchi.

\chapter{Driver a carattere}
\section{Introduzione}
I \textbf{device driver} (gestori delle periferiche) sono dei moduli software
che adattano una l'hardware di una specifica periferica a funzionare nel
contesto del sistema operativo. I gestori sono suddivisi in \textbf{character
devie driver} e \textbf{block device driver}.

\section{File speciali e identificazione dei gestori}
Per ogni periferica installata nel sistema deve esistere un corrispondente file
speciale (di solito nella cartella \texttt{/dev}).

Ogni periferica è identificata da una coppia di numeri: \textbf{major} e
\textbf{minor}. Il primo identifica il driver, il secondo il dispositivo.
Questi valori sono posti nell'inode del file speciale.

\section{Routine di un gestore di periferica}
Le funzioni che un gestore di periferica svolge sono:
\begin{itemize}
	\item inizializzazione all'avvio del SO
	\item porre la periferica in servizio o fuori servizio
	\item ricevere e inviare dati
	\item scoprire e gestire gli errori
	\item gestire gli interrupt
\end{itemize}
Il gestore è quindi costituito da routine attivate al momento opportuno e dalla
routine di interrupt. In ogni gestore esiste una \textbf{tabella delle
operazioni} così composta:
\begin{verbatim}
struct file_operations{
    int (*lseek) ();
    int (*read) ();
    int (*write) ();
    int (*ioctl) ();
    int (*open) ();
    int (*release) ();
    ...
}
\end{verbatim}
\begin{itemize}
	\item \texttt{open()} verifica che la periferica sia attiva, e nel caso
		questa si utilizzabile da un solo processo alla volta viene controllata
		un'apposita variabile (e nel caso viene restituito un errore).
	\item \texttt{release()} viene chiamata dalla \texttt{close()} di un file
		speciale: controlla che il buffer sia svuotato e modifica la variabile
		di utilizzo.
	\item \texttt{write()}: vedi \ref{write}
	\item \texttt{read()} simmetrica a \texttt{write()}.
	\item \texttt{ioctl(int fd, int command, int parameter)} permette di
		svolgere le operazioni specifiche della periferica.
\end{itemize}
All'avvio del SO viene attivata una funzione di inizializzazione per ogni
gestore di periferica installato che restituisce un puntatore alla propria
tabella delle operazioni. Il kernel si salva il puntatore nella \textbf{tabella
dei gestori a carattere} o nella \textbf{tabella dei gestori a blocchi}.

Ad esempio, nel caso di una periferica a carattere (che non utilizza la page
cache), il kernel accede al file \texttt{/dev/tty1}, dove legge il numero di
major \texttt{4} , per accedere alla tabella delle operazioni e chiamare la
funzione \texttt{open()}.

\section{Principi di funzionamene di un gestore di periferica}
Essendo che un processo in attesa di un servizio della periferica non è in
esecuzione, questo non è in grado di rispondere prontamente alla richiesta di
interrupt e di trasferire i dati. Perciò i dati da scrivere o leggere vengono
conservati dal gestore stesso in un buffer.
\begin{itemize}\label{write}
	\item il processo chiama X.write()
	\item la routine di X.write() copia dallo spazio del processo nel buffer
		del gestore $C = min(buffer.size(), char_to_write)$ caratteri. 
	\item X.write() manda il primo carattere alla periferica.
	\item X.write() deve attendere che la stampante abbia stampato il carattere
		e sia pronta a ricevere il successivo.
		\texttt{wait\_event\_iterruptible(WQ, empty\_buffer)}.
	\item interrupt della periferica
	\item la routine di interrupt viene subito eseguita e se disponibile il
		carattere successivo viene inviato, altrimenti...
	\item \texttt{wake\_up(WQ)} risveglia il processo e lo mette in stato di
		pronto.
	\item quando il processo riprende l'esecuzione X.write() copia i successivi
		caratteri nel buffer del gestore, altrimenti...
	\item il processo ritorna al modo U.
\end{itemize}

\end{document}
