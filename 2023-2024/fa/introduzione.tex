\documentclass[12pt, a4paper]{article}

\usepackage[margin=2cm]{geometry}
\usepackage{ragged2e}
\usepackage{amssymb}
\usepackage{amsmath}
\usepackage{amsthm}
\usepackage{hyperref}
\usepackage{thmtools}
\usepackage[utf8]{inputenc}
\usepackage{tocloft}
\usepackage{mathrsfs}

\renewcommand{\listtheoremname}{Indice}
\newtheoremstyle{break}
	{}           %Space above, empty = usual value
	{}           %Space below
	{}           %Body font
	{}           %Indent amount (empty = no indent, \parindent = para indent)
	{\bfseries}  %Thm head font
	{.}          %Punctuation after thm head
	{\newline}   %Space after thm head: \newline = linebreak
	{}           %Thm head spec
\renewcommand{\proofname}{Dimostrazione}
\theoremstyle{break}

\newtheorem{theorem}{Teorema} %[section]
\newtheorem*{definition}{Definizione}
\newtheorem*{exercise}{Esercizio}
\newtheorem*{example}{Esempio}

\title{Introduzione}
\author{Gabr1313}
\date{\today}

\begin{document}
\maketitle
\tableofcontents
\justify
\sloppy

\newpage
\section{Info}
Laboratori 2 e 23 Maggio (gli altri giovedì sono liberi)
Di solito le esercitazioni sono il lunedì

\section{Sistema}
L'automatica è insieme strumenti che permettono di risolvere un problema di 
controllo, modelizzati attraverso un modello matematico.
\begin{definition}[Sistema di controllo] Un sistema di controllo è composto 
    da:
    \begin{itemize}
        \item Variabili in ingresso $\bar{u}$
            \begin{itemize}
                \item manipolabili
                \item distrubi
            \end{itemize}
        \item Variabili in uscita $\bar{y}$
            \begin{itemize}
                \item controllate
                \item misure
            \end{itemize}
        \item Modello: comportamento del sistema (ingressi $\to$ uscite) \\
              meglio se non varia nel tempo
    \end{itemize}
\end{definition}
\begin{definition}[Retroazione]
    Misurare, controntare la misura con un \textbf{set-point}, aggiustare i
    parametri in ingresso
\end{definition}
\begin{definition}[Legge di controllo]
    Imporre alle variabili un comportamento: può essere fatto manualmente, o 
    automaticamente.
\end{definition}
\begin{definition}[Attuatori]
    Strumenti che aiutano a modificare le variabili manipolabili
\end{definition}
\begin{definition}[Sensori / Trasduttori]
    Strumenti che aiutano a misurare le varibali di uscita
\end{definition}
\begin{definition}[Leggi di controllo]
    Viene applicata attraverso un HW (calcolatore), che interagisce con
    attuatori e trasduttori
\end{definition}

\begin{example} [Resistore] 
    In un resistore...
    \begin{itemize}
        \item $u = i$
        \item $y = v$
        \item modello $y = Ru$
    \end{itemize}
\end{example}

\end{document}

